\chapter{Conclusion}

This project aimed to fulfil a gap that was present inside almost every other nutrition application - 
giving nutritional information about restaurant's meals while providing insulin dosages calculations
to users with type 1 diabetes.\\

To this end, two client applications, a HTTP server and a database were developed.
The platform's backend provided the clients the most important capability that was accomplished by the end
of the project - platform's self-maintainability.\\

This mentioned characteristic is benevolent when developing a highly scalable platform like this one,
where posterior centralized data validations becomes inviable due to large amounts of generated information.\\

Reached this part of the report, it can be concluded that every functional and non-functional requirement which 
was initially planned was accomplished by the end of this project. Some considerations about the project's future
development were also discussed by the group and are presented in the next section.\\

\section{Future development}

As this project represents a final bachelor's project, some future work could still be done in order
to publish and deploy the platform to the general public, making it more successful and profitable.\\

Here are some topics we agreed that could belong to this project's future development:

\subsubsection{Statistics analysis}

Given that one of this project's main strengths is information storage and mapping, we found
relevant that statistics analysis about submitted restaurants and meals should be done, 
in order to help and provide other work fields information that could be useful.\\

However every data that is collected should be properly disclaimed and only the information that
would not compromised the user's privacy should be colectable.

\subsubsection{Interaction with similar platforms}

As written in the last topic, the information could enrich the world of nutrition and health platforms.
To allow this, the API could be available to the public, so it could be integrated with other platforms
alike.\\

\subsubsection{Improved social system}

This platform depends strongly on the community, as such there should be a social system improval in order
to make its members more active and the platform's more responsive and fulfilled.\\

One way to do that could be social rewards for the most active and contributive users.

\subsubsection{Client UI/UX revision}

As the group that developed this platform is not qualified in this matter, 
the user interface should be reviewed and improved by an UI/UX team, which has the tools to make the clients' 
interface more appealing to the users.

\subsubsection{Certified members}

Anyone can participate and make submissions to the platform. Given this fact, voided information
can be submitted and spread inside the community, contributing to misinformation.\\

To tackle this issue there should be member certifications for the most active members or even
certified professionals, which could validate submitted information and make the platform's
environment a more trustworthy place online. 

\subsubsection{Two-factor authentication}

Security is a very important topic nowadays, being a field that is constantly evolving. As it is proven
that single-factor authentication might be not enough to avoid attacks completely, a two-factor authentication
could be implemented to improve user's security.

\subsubsection{Email verification}

Speaking of security, an element required for two-factor authentication would be email verification. This would not be 
only useful for this feature, as it would provide the user the ability to recover its password, avoid bots from registering
in our platform and provide a more safer user account removal, as the actual account removal verifies only the passed
JSON Web Token to remove it. 