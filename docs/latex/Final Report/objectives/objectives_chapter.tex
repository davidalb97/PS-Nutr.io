\chapter{Project's context and objectives}

\section{Context}

Type 1 diabetes is caused by an autoimmune reaction where the body’s defence system attacks the cells that produce insulin. 
As a result, the body produces very little or no insulin. 
The exact causes of this are not yet known, but are linked to a combination of genetic and environmental conditions.
\cite{idf}\\

Healthy nutrition — knowing when and, most importantly, what to eat — is an important part of 
diabetes management as different foods affect your blood glucose levels differently. Foods with a 
carbohydrate count require an insulin dosage to be administered which is calculated based 
on an uniquely crafted insulin profile for the person with type 1 diabetes.\\

Knowing the carbohydrates of what is being eaten solely relies on mapping a food's portion (such as weight or cups) to its' nutritional 
value using an official nutritional sheet \cite{mellitus} or a nutritional application, 
meaning that with access to a food scale this task becomes simple. However, it is unrealistic to expect a person with type 1 diabetes
to always bring a food scale to every scenario - such as a restaurant - leading to inaccurate estimations on eaten meals and as such,
incorrect insulin dosages.\\

Given that one of the group members has type 1 diabetes and that according to IDF, an estimated 1.1 million children and adolescents under the age of 20 
live with type 1 diabetes \cite{idfatlas} the group decided to focus on this subject
and as such faced a problem:\\ 

Despite the fact that countless nutritional applications are capable of providing nutritional information, it is not  
accurate and trustworthy regarding complex meals and even when it is, this information is always isolated in context, 
meaning that said meal always has a generic portion and is served in a generic place.\\

\newpage
\section{Real world comparision}

In order to consolidate and define the problem in relation to the project the group
researched various mobile nutritional applications, amongst them "MyFitnessPal" as the main comparison.
This application allows users to search a meal and its' information by name, recipe, commercial barcode or geolocation. The latter
only providing meals for popular fast food chains around the user.\\

The nutritional book and mobile application "Carbs and Cals" is also worth mentioning as it also provides pictures 
of searched meals, allowing the user to better understand and visualize the portions that given meal might have.

\section{Objectives}

\begin{itemize}
    \item Design a system that helps individuals with type 1 diabetes easing difficult carbohydrate measurement situations, specifically in restaurants.
    \item Build a platform maintained by its community, using users' submissions to improve the data's accuracy;
    \item Deliver a mobile application where the user can search nearby restaurants and their meals;
    \item Design an insulin calculator that computes insulin dosages based on the nutritional information of the meals selected by the user;
    \item Protect user's sensitive data, such as insulin profiles, via encryption.
\end{itemize}