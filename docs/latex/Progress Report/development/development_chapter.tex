%
% Chapter 2 - Report Development
%
\chapter{Project development}
    
	\section{Roadmap}

    According to the proposed plan, the group managed to fulfill almost all the objectives
    that were planned to be accomplished by this time - a relational database, a working HTTP server
    and a prototype Android aplication that can make requests to the server and 
    use its information to display lists and detailed views.\\

    However, as it will be detailed in the next section, the group faced some issues that caused slight
    drawbacks, such as the relational database conception, which had to be redesigned multiple
    times due to the project's requirements.\\

    In this report's appendix there are disposed two project schedules: the initial one [\nameref{app:initial_plan}], also present in the
    project's proposal, and the actual one [\nameref{app:actual_plan}], which has the actual progress made to this date with the corrects starting
    and ending dates of each concluded milestone.


    \section{Issues encountered and updates}

    This section describes the issues found and the decisions made
    to overcome them, during the first project modules' development.
    
    \subsection{Relational database}

    The group had to redesign the database's conceptual model and relational model multiple times.
    One of the main issues that supported the main redesign was that the database 
    only supported submissions from the users, while it had to support submissions both from users and APIs.\\

    The other redesigns were related to the database normalization: it was normalized until the third normal form 
    and had to be renormalized one more time because of the mentioned main redesign.\\ 

    Because of this, the group invested a considerable amount of time on the database's conception, which was not considered
    in the initial plan. This invested time ended up colliding partially with other time periods that were
    reserved for other modules.

    \subsection{HTTP server}

    As mentioned in the project's proposal risks, the HTTP server is very similar to the DAW's project, which is an optional
    course inside the LEIC programme. Thus some improvements were made and errors were fixed because of the lectioned classes 
    and the first project's delivery in DAW, which helped the group complete the tasks implied by the server's development.\\

    However the server is lacking some functionalities as a result of the time invested in database conception,
    where the HTTP server depends strongly, causing some drawback to its development.

    \subsection{Android application}

    As a result of the time invested developing the database and the strong bond between the mobile application and the HTTP server 
    and its completion, the Android app suffered a slight drawback, where only the essential fragments and requests work.\\

    Althrough it was initially planned that the mobile application was just a prototype by this date, as it is now, the group
    could not implement more planned features because of the time invested in the other modules.

    \section{Group decisions}

    Given the previous issues, the group has the conditions to comply with the project's initial plan, with some minor changes.\\
    
    As stated in the last period of the Android application's subsection, the group strongly agrees that the HTTP server has to be finished
    as soon as possible, to provide stability to the onward modules and their implementations, being considered a core module in this project.\\

    Subsequently the group will have to invest a few more days in the server's development in order to fulfill the project's requirements.\\

    Alongside with these fulfillments, the group will proceed to add more features to the mobile application, while starting the web application's development,
    which is compatible with the project's initial plan.