%
% Chapter 2 - Report Development
%
\chapter{Project development}
    
	\section{Roadmap}

    According to the proposed plan, the group managed to fulfill almost all the objectives
    that were planned to be accomplished until now - a relational database, a working HTTP server
    and a prototype Android aplication that can make requests to the server and 
    use its information to display lists and detailed views.\\

    However, as it will be detailed in the next section, the group faced some issues that caused slight
    drawbacks, such as the relational database conception, which had to be redesigned multiple
    times due to the project's requirements.\\

    This report's appendix displays two project schedules: the initial one [\nameref{app:initial_plan}], also present in the
    project's proposal, and the actual one [\nameref{app:actual_plan}], which was updated to match the 
    actual progress made until now.


    \section{Issues encountered and updates}

    This section describes the issues found and the decisions made
    to overcome them during development.
    
    \subsection{Relational database}

    The group had to redesign the database's models multiple times.
    One of the main issues that drove the redesign was the fact that the previous implementation 
    only supported submissions from users, while it had to support submissions from both users and APIs.\\

    Another issue is that our model offered no support to distinguish between API types and their submission, meaning that,
    for example, different restaurants from two different APIs (e.g.: Zomato and Yelp) with equal identifiers would generate data collisions.\\

    It is worth mentioning that with every major addition, the database had to be renormalized, taking a considerable amount of time,
    which was not taken into account in the initial plan.\\

    As such, this invested time ended up partially colliding with other schedules that were
    reserved for other modules.

    \subsection{HTTP server}

    The group initially struggled to implement the HttpServer as we were utilizing and learning a new framework - Spring MVC, meaning that implementing
    known patterns in server programing took more time than expected in the according to the initial project plan.\\

    The group also recognizes that some functionalities are lacking as a result of the time invested in the database, namely endpoints that allow for resource creation.

    \section{Roadmap updates}

    Given that the previously mentioned issues are only minor setbacks, the group agrees that every mandatory requirement made in the initial draft can still be implemented,
    however, the group expresses concerns that optional requirements will have to be reevaluated after the Beta release, if needed.