%
% Chapter 0 - Abstract
%

\chapter*{Abstract}
    The idea that every field of study can be digitalized in order to ease 
    monotonous tasks is continuously growing in the modern world. Our project
    aims to tackle the field of Type 1 diabetes, given its growing prevalence in the world.\\

    One of those monotonous tasks is the count and measurement of carbohydrates in meals used to
    administer the correspondent amount of insulin, along with their blood levels, to maintain a healthy lifestyle.
    A task that heavily relies on having access to food databases and realize of how many portions a meal has - usually
    by using a digital balance or doing estimations.\\

    Eating in restaurants is the perfect example that showcases a gap in this field, that our project, 
    Nutr.io, aims to fill.  Most nutritional applications do not provide data for restaurants' meals,
    such as MyFitnessPal, nor does the user bring his digital balance from home - resulting in a faulty
    carbohydrate count and therefore the administration of an incorrect insulin dose.\\

    The main goal of this project is to design a system that offers a way to facilitate difficult
    carbohydrate measurement situations, like in restaurants.
    To that end, a system that stores meals' nutritional information will be developed, 
    where users can use and calibrate its data with their feedback.\\