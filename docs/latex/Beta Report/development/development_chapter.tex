%
% Chapter 2 - Report Development
%
\chapter{Project development}
    
	\section{Roadmap}

    According to the proposed plan, the group managed to accomplish a more complete Android aplication 
    that can make requests to the server and use its information to display lists,
    detailed views and calculate user's insulin values as store local cache such as 
    user profiles and search history.\\

    The HTTP server also had great progress, having now most of its planned features working and properly tested. In order
    to make this possible, the database suffered some changes that will be detailed later in this report.\\

    However, as it will be detailed in the next section, the group faced some issues that caused
    delays and made the web browser application development not possible in this delivery, mainly due to API
    withdrawals and data models changes.\\

    \section{Issues encountered and updates}

    This section describes the issues found and the decisions made
    to overcome them during development.
    
    \subsection{Relational database}

    The group had to redesign the database's models multiple times again, due to the previously mentioned API withdrawals and other encountered
    incoherences.\\    

    \subsection{Food API's}
    Throughout development and with some research, the group concluded that no meal API exists
    that provides accurate nutritional - either for a recipe/meal or a base ingredient, such as rice.\\

    This conclusion came from investigating three possible API's: Edamam, Nutrionix and Spoonacular; and comparing
    their provided nutritional values with corresponding values obtained from
    certified sources for Portugal. The results can be found in TODO MIGUEL.

    The group assumes that this inaccuracy is due to the fact that
    mentioned API's 
    \subsection{HTTP server}

    After the progress report, the group continued developing the HTTP server and completing the endpoints that were previously lacking. However
    it was concluded that most APIs that were being used by the server didn't provide accurate nutritional information about meals and ingredients,
    which would void the main objective of our platform.\\

    As a result, the group had to drop some meal APIs and introduce generic and hardcoded meal information inside the database as a replace.\\

    This information will be progressively tuned along with the users inputs when they, for example, associate a meal with a restaurant, leaving
    their values about meals' quantities and portions.

    \subsection{Android client}

    The Android application's development progressed normally but it had to be put on hold sometimes, because of HTTP server's endpoints' completion, in which the application
    depends strongly. A major dto and model restructure had also to be made inside the mobile application in order to meet with the current HTTP responses.\\

    \section{Roadmap updates}

    Given that the previously mentioned issues, the group agrees that every mandatory requirement made in the initial draft
    and retified in the previous progress report can still be implemented until the project's final release, such as the web client apllication.\\
    
    However the group will have to remove some optional features so the main ones can be delivered whole.